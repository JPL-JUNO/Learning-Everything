\chapter{Echarts + Vue}
\section{图标相关的基本配置}
\begin{itemize}
    \item \verb|xAxis|: 直角坐标系 中的 x 轴, 如果 type 属性的值为 category ,那么需要配置 data 数据, 代表在 x 轴的呈现
    \item \verb|yAxis|: 直角坐标系 中的 y 轴, 如果 type 属性配置为 value , 那么无需配置 data , 此时 y 轴会自动去 series 下找数据进行图表的绘制
    \item \verb|series|: 系列列表。每个系列通过 type 决定自己的图表类型, data 来设置每个系列的数据
\end{itemize}

更多配置项可以参考\href{https://echarts.apache.org/zh/option.html#title}{Echarts官方文档}。

\section{基础图标}
\subsection{柱状图}
常见的效果:
\begin{itemize}
    \item 标记(最大值markPoint、最小值markPoint、平均值markLine)
    \item 显示(数值显示label、柱宽度barWidth、横向柱状图)
\end{itemize}
\subsection{通用配置}
\paragraph{标题title} 用于设置图标的标题,主要包括标题文字、颜色、边框、边框圆角、位置
\paragraph{提示框tooltip} 用于配置鼠标或点击图标时的显示框。
\begin{enumerate}
    \item 触发类型:trigger,主要用两个可选值:item、axis
    \item 触发时机:triggerOn
    \item 格式化:formatter
\end{enumerate}
\paragraph{工具按钮toolbox} Echarts提供的工具栏:内置有导出图片saveAsImage、数据视图dataView、动态类型切换magicType、数据区域缩放dataZoom、重置restore五个工具。

\begin{itemize}
    \item legend 中的 data 是一个数组
    \item legend 中的 data 的值需要和 series 数组中某组数据的 name 值一致
\end{itemize}
\paragraph{图例legend} 图例,用于筛选系列,需要配合series使用

\subsection{折线图}
常见效果:
\begin{itemize}
    \item 标记(最大值markPoint、最小值markPoint、平均值markLine、标注区间markArea)
    \item 线条控制(平滑smooth、样式lineStyle)
    \item 填充风格areaStyle、紧挨边缘boundaryGap
    \item 缩放脱离0值的比例
\end{itemize}

\subsection{散点图}
数据应该是一个二维数组

常见效果:
\begin{itemize}
    \item 气泡图效果:散点的大小symbolSize、散点的颜色itemStyle.color
    \item 涟漪动画效果:\verb|type: effectScatter|, \verb|showEffectOn: "emphasis"|, \verb|rippleEffect|
\end{itemize}
\subsection{直接坐标系的常用配置}
这叫坐标系的常见图表: 柱状图、折线图、散点图。

\begin{enumerate}
    \item 网格grid: show, borderWidth, borderColor, top, left, right, bottom, height, width;
    \item 坐标轴: 坐标轴分为x轴和y轴, 一个 grid 中最多有两种位置的 x 轴和 y 轴.
    \begin{itemize}
        \item 坐标轴类型 type: value : 数值轴, 自动会从目标数据中读取数据, 如果是category : 类目轴, 该类型必须通过 data 设置类目数据
        \item 坐标轴位置: xAxis :  ["top", "bottom"], yAxis : ["left", "right"]
    \end{itemize}
    \item 区域缩放dataZoom:用与区域缩放,对数据范围进行过滤,$x$轴和$y$轴都可以拥有,dataZoom接收一个数组,因此可以配置多个区域缩放器。
    \begin{itemize}
        \item 类型type:slider:滑块,inside,默认,依靠鼠标滚轮或者双指缩放;
        \item 指标产生作用的轴:xAxisIndex:设置缩放组件控制的是哪个x轴,一般写0即可,yAxisIndex同理;
        \item 指明初始状态的缩放情况:start:数据窗口范围的只是百分比,end:数据窗口范围的结束百分比;
    \end{itemize}
\end{enumerate}

\subsection{饼图}
饼图的数据是由 name 和 value 组成的字典所形成的数组,饼图无须配置 xAxis 和 yAxis。

常见效果:
\begin{itemize}
    \item 显示数值:\verb|label.formatter|
    \item 圆环:设置两个半径~\verb|radius: ["50%", "70%"]|
    \item 南丁格尔图:南丁格尔图指的是每一个扇形的半径随着数据的大小而不同, 数值占比越大, 扇形的半径也就越大~\verb|roseType: "radius"|
    \item 选中效果: 
    \begin{itemize}
        \item \verb|selectedMode: ["multiple", "single"]|
        \item \verb|selectedOffset: 30|
    \end{itemize}
\end{itemize}

\subsection{地图}
\begin{enumerate}
    \item ECharts 最基本的代码结构
    \item 准备中国的矢量 json 文件, 放到 json/map/ 目录之下
    \item 使用 Ajax 获取 china.json
    \item 在Ajax的回调函数中, 往 echarts 全局对象注册地图的 json 数据 echarts.registerMap('chinaMap', chinaJson)
    \item 获取完数据之后, 需要配置 geo 节点, 再次的 setOption
\end{enumerate}

常见配置:缩放拖动: roam,名称显示: label,初始缩放比例: zoom,地图中心点: center

常见效果:
\begin{itemize}
    \item 显示某个区域:
    \begin{enumerate}
        \item 准备安徽省的矢量地图数据
        \item 加载安徽省地图的矢量数据
        \item 在Ajax的回调函数中注册地图矢量数据
        \item 配置 geo 的 type:'map' , map:'anhui'
        \item 通过 zoom 调整缩放比例
        \item 通过 center 调整中心点
    \end{enumerate}
    \item 不同城市显示不同颜色
    \begin{enumerate}
        \item 显示基本的中国地图
        \item 准备好城市空气质量的数据, 并且将数据设置给 series
        \item 将 series 下的数据和 geo 关联起来
        \item 结合 visualMap 配合使用:visualMap 是视觉映射组件, 和之前区域缩放 dataZoom 很类似, 可以做数据的过滤. 只不过 dataZoom 主要使用在直角坐标系的图表, 而 visualMap 主要使用在地图或者饼图中
    \end{enumerate}
    \item 地图和散点图结合
\end{itemize}
\paragraph{地图和散点图结合}
\begin{enumerate}
    \item 给 series 这个数组下增加新的对象
    \item 准备好散点数据,设置给新对象的 data
    \item 配置新对象的 type:\verb|type:effectScatter|
    \item 让散点图使用地图坐标系统\verb|coordinateSystem: "geo"|
    \item 让涟漪的效果更加明显
\end{enumerate}

\subsection{雷达图}
\paragraph{实现步骤}
\begin{enumerate}
    \item ECharts 最基本的代码结构: 引入JS文件、DOM容器、初始化对象、设置Option
    \item 定义各个维度的最大值:
    \item 准备具体产品的数据:
    \item 
\end{enumerate}

\paragraph{常用配置}
\begin{itemize}
    \item 显示数值:\verb|label.show: true|
    \item 区域面积:\verb|reaStyle: {}|
    \item 绘制类型:\verb|shape: "circle"|
\end{itemize}

\subsection{仪表盘}
仪表盘主要使用在进度把控以及数据范围的监控。
\paragraph{实现步骤}
\begin{enumerate}
    \item ECharts 最基本的代码结构: 引入JS文件、DOM容器、初始化对象、设置Option
    \item 准备数据, 设置给 series 下的 data
    \item 在 series 下设置 type:gauge
\end{enumerate}
\paragraph{常用效果}
\begin{itemize}
    \item 显示数值:\verb|label.show: true|
    \item 区域面积:\verb|reaStyle: {}|
    \item 绘制类型:\verb|shape: "circle"|
\end{itemize}

\section{ECharts高级}
\subsection{显示相关}
\subsubsection{主题}
ECharts 中默认内置了两套主题: light、dark,在初始化对象方法 init 中可以指明。
\begin{js}
var myChart = echarts.init(document.querySelector("div"), "light");
var myChart = echarts.init(document.querySelector("div"), "shine");
\end{js}
\subsubsection{调色板}
它是一组颜色,图形、系列会自动从其中选择颜色, 不断的循环从头取到尾, 再从头取到尾, 如此往复。有主题调色盘、全局调色盘(将覆盖主题调色盘)、局部调色盘(将覆盖全局调色盘)

颜色渐变:线性渐变itemStyle、径向渐变
\begin{js}
color: {
    type: "radial", // 径向渐变
    // x, y, r都为相对值
    x: 0.5,
    y: 0.5,
    r: 1,
    colorStops: [
        { offset: 0, color: "red" },
        { offset: 1, color: "green" },
    ],
},
\end{js}
\subsubsection{样式}
直接样式(itemStyle, textStyle, lineStyle, areaStyle, label)

高亮样式(emphasis中包裹itemStyle, textStyle, lineStyle, areaStyle, label):图表中, 其实有很多元素都是有两种状态的, 一种是默认状态, 另外一种就是鼠标滑过或者点击形成的高亮状态. 而高亮样式是针对于元素的高亮状态设定的样式。
\subsubsection{自适应}
当浏览器的窗口大小发生变化的同时,需要图表的大小也随之适配变化。

\begin{enumerate}
    \item 监听窗口大小变化事件
    \item 在事件处理函数中调用 ECharts 实例对象的 resize 即可
\end{enumerate}
\begin{js}
window.onresize = function () {
myChart.resize();
};
\end{js}
\subsection{动画的使用}

\subsubsection{加载动画}
ECharts 已经内置好了加载数据的动画, 我们只需要在合适的时机显示或者隐藏即可。

一般, 我们会在获取图表数据之前显示加载动画,主要是因为后台获取数据可能会需要一定的时间。
\begin{itemize}
    \item 显示加载动画\verb|myChart.showLoading()|
    \item 隐藏加载动画\verb|myChart.hideLoading()|
\end{itemize}

\subsubsection{增量动画} setOption可以设置多次,新的option和旧的option的关系并不是相互壶盖的区别,是相互整合的关系,在设置新的option的时候,只需要考虑到变化的部分就可以了。

所有数据的更新都是通过setOption实现的,不需要考虑数据的具体变化位置,ECharts会找到两组数据之间的差异,然后通过合适的动画去表现数据的变化。
\subsubsection{动画的配置} 
\begin{itemize}
    \item 开启动画,\verb|animation: true|
    \item 动画时长(毫秒单位):\verb|animationDuration: 7000|,也可以接收一个回调参数
    \item 缓动动画:\verb|animationEasing: 'linear'|
    \item 动画阈值:\verb|animationThreshold: 10|,单种形式的元素数量大于这个阈值时会关闭动画
\end{itemize}
\subsection{交互API}
\subsubsection{全局echarts对象}
全局echarts对象是引入echarts.js文件之后就可以直接使用的。

\begin{itemize}
    \item init:初始化echarts实例对象,可以用来使用主题
    \item registerTheme:
    \item registerMap
    \item connect
\end{itemize}
\subsubsection{echartsInstance对象}
echartsInstance对象是通过\verb|echarts.init|方法调用之后得到的。
