\chapter{Vue基础}
\section{初识Vue}

\section{模板语法}
Vue模板语法有2大类:
\begin{enumerate}
    \item 插值语法:用于解析标签体内容,主要写法是:\verb|{{xxx}}|,\verb|xxx|是JS表达式,且可以直接读取到data中的所有属性。
    \item 指令语法:用于解析标签(包括:标签属性、标签体内容、绑定事件......)。例如:\verb|v-bind:href="xxx"|,\verb|xxx|是JS表达式,且可以直接读取到data中的所有属性。
\end{enumerate}
\section{数据绑定}

Vue中有两种数据绑定方式:
\begin{enumerate}
    \item 单向数据绑定(v-bind):数据只能从data流向页面。
    \item 双向数据绑定(v-model):数据不仅能从data流向页面,还可以从页面流向data。双向绑定一般都应用在表达类元素上,如input、select等,\verb|v-model: value|可以简写为:\verb|v-model|,因为\verb|v-model|默认收集的就是value的值。
\end{enumerate}
\begin{html}
<div id="root">
    <input type="text" v-bind:value="name"><br />
    <input type="text" v-model:value="name">
    <input type="text" v-model="name">
</div>
\end{html}

\section{el与data的两种写法}
\begin{itemize}
    \item el有2种写法:
    \begin{enumerate}
        \item new Vue配置el属性
        \item 先创建Vue实例,然后通过\verb|vm.$mount('#root')|指定el的值。
    \end{enumerate}
    \item data有2种写法:
    \begin{enumerate}
        \item 对象式
        \item 函数式
    \end{enumerate}
\end{itemize}

一个重要的原则:由Vue管理的实例,一定不要写箭头函数,一旦写了箭头函数,this指针就不再是Vue实例了。

\section{理解MVVM模型}

\section{事件处理}
事件的基本使用:
\begin{enumerate}
    \item 使用\verb|v-on:xxx| 或 \verb|@xxx|绑定事件,其中\verb|xxx|是事件名;
    \item 事件的回调需要配置在methods对象中,最终会在vm上;
    \item methods中配置的函数,不要用箭头函数!否则this就不是vm了(变成了window);
    \item methods中配置的函数,都是被Vue所管理的函数,this的指向是vm 或 组件实例对象;
    \item \verb|@click="demo"| 和 \verb|@click="demo($event)"| 效果一致,但后者可以传参;
\end{enumerate}

\subsection{事件修饰符}
主要由以下6种事件修饰符,用于操作对事件的处理。
\begin{enumerate}
    \item prevent:阻止默认事件(常用);
    \item stop:阻止事件冒泡(常用);
    \item once:事件只触发一次(常用);
    \item capture:使用事件的捕获模式;
    \item self:只有event.target是当前操作的元素时才触发事件;
    \item passive:事件的默认行为立即执行,无需等待事件回调执行完毕;
\end{enumerate}

事件修饰符可以连续写,比如\verb|@click.stop.prevent|,先阻止冒泡,然后阻止默认事件。

\subsection{键盘事件}
Vue中常用的按键别名:回车(enter)、删除(delete, 捕获“删除”和“退格”键)、退出(esc)、空格(space)、换行(tab, 特殊,必须配合keydown去使用)、上(up)、下(down)、左(left)、右(right)。

Vue未提供别名的按键,可以使用按键原始的key值去绑定,但注意要转为kebab-case(短横线命名)。

系统修饰键(用法特殊):ctrl、alt、shift、meta
\begin{enumerate}
    \item 配合keyup使用:按下修饰键的同时,再按下其他键,随后释放其他键,事件才被触发。
    \item 配合keydown使用:正常触发事件。
\end{enumerate}

此外,也可以使用keyCode去指定具体的按键(不推荐),\verb|Vue.config.keyCodes.customName = keyCode|,可以去定制按键别名。

\begin{html}
<div id="root">
    <h1>Hello, {{ name }}</h1>
    <input type="text" placeholder="enter for hint information" @keyup.enter="showInfo">
</div>
\end{html}

\begin{html}
<script>
    new Vue({
        el: "#root",
        data: {
            name: 'Vue2-3'
        },
        methods: {
            showInfo(event) {
                // if (event.keyCode !== 13) return
                console.log(event.target.value)
            }
        }
    })
</script>
\end{html}
\section{计算属性与侦听属性}
\subsection{计算属性}
需要使用的属性不存在,要通过vm实例已有的属性(Property)计算得来,底层借助了\verb|Object.defineproperty|方法提供的\verb|getter|与\verb|setter|。

get函数执行的时刻为以下2种:
\begin{enumerate}
    \item 初次读取时会执行一次;
    \item 当依赖的数据发生改变时会被再次调用;
\end{enumerate}

尽管使用methods方式以及插值方式都可以实现,但是计算属性由内部缓存机制(复用),调试方便。

\begin{enumerate}
    \item 计算属性最终会出现在vm上,直接读取使用即可。
    \item 如果计算属性要被修改,那必须写set函数去响应修改,且set中要引起计算时依赖的数据发生改变。
\end{enumerate}

\begin{js}
computed: {
    fullName: {
        get() {
            return this.firstName + '-' + this.lastName
        },

        set(value) {
            console.log('set', value);
            const arr = value.split('-');
            this.firstName = arr[0];
            this.lastName = arr[1];
        }
    }
}
\end{js}

如果不考虑修改计算属性,那么get的计算属性可以简写为:
\begin{js}
computed: {
    fullName() {
        return this.firstName + ' ' + this.lastName
    }
}
\end{js}
\subsection{侦听属性}
侦听属性(watch),当被侦听的属性变化时, 回调函数自动调用, 进行相关操作,侦听的属性必须存在,才能进行侦听。侦听有2种写法:
\begin{enumerate}
    \item new Vue时传入watch配置
    \begin{js}
watch: {
    isHot: {
        immediate: true,
        handler(newValue, oldValue) {
            console.log('isHot was modified', newValue, oldValue)
        }
    }
}
    \end{js}
    \item 通过\verb|vm.$watch|侦听
    \begin{js}
vm.$watch('isHot', {
    immediate: true,
    handler(newValue, oldValue) { 
        console.log('isHot was modified', newValue, oldValue)
    }
})      
    \end{js}
\end{enumerate}

Vue中的watch默认不监测对象内部值的改变(一层、最外层),配置\verb|deep: true|可以监测对象内部值的改变(内层)。

Vue自身可以监测对象内部值的改变,但Vue提供的watch默认不可以,使用watch时根据数据具体结构,决定是否使用深度监测。

侦听多层级中某个属性的变化:
\begin{js}
"numbers.a": {
    handler() {
        console.log('a was modified')
    }
}
\end{js}

侦听层级中任一属性值的改变:
\begin{js}
numbers: {
    deep: true,
    handler() {
        console.log('numbers were modified')
    }
}
\end{js}

如果没有深层次的参数,那么侦听函数可以简写:
\begin{js}
isHot(newValue, oldValue) {
    console.log('isHot was modified', newValue, oldValue)
},
\end{js}

computed与watch的对比:
\begin{enumerate}
    \item computed能完成的功能,watch都能完成
    \item watch能完成的功能,computed不一定能够完成,比如watch可以执行异步操作,而computed不能执行。
\end{enumerate}

两个重要的原则:
\notes{
    \begin{enumerate}
        \item 所有被Vue管理的函数,最好写成普通函数,这样this的指向才是vm或组件实例对象;
        \item 所有不被Vue管理的函数(定时器的回调函数、Ajax的回调函数等),最好写成箭头函数,这样this的指向才是vm或者组件实例对象。
    \end{enumerate}
}

\section{绑定样式}

\subsection{绑定class样式}
字符串写法:样式的类名不确定,需要动态指定,但是其个数是确定的,仅追加一个样式值。
\begin{html}
<div class="basic" :class="mood" @click="changeStyle">{{ mood  }}</div>
\end{html}

数组写法:要绑定的样式个数与名称均不确定。
\begin{html}
<div class="basic" :class="moodArr">{{ mood }}</div>
\end{html}

对象写法:要绑定的样式个数确定、名字也确定,但是动态决定用不用。
\begin{html}
<div class="basic" :class="moodObj">{{ mood }}</div>
\end{html}
\subsection{绑定style样式}
可以使用对象或者数组进行样式的绑定,但是实际中使用的不多。

\begin{html}
<div class="basic" :style="styleObj">{{ mood }}</div>
<br>
<div class="basic" :style="[styleObj, styleObj2]">{{ mood }}</div>
<br>
<div class="basic" :style="styleArr">{{ mood }}</div>
\end{html}

\begin{js}
fontSize: 40,
    styleObj: {
        fontSize: '40px',
        color: 'red',
    },
    styleObj2: {
        backgroundColor: 'orange'
    },
    styleArr: [
    {
        fontSize: '40px',
        color: 'red',
    },
    {
        backgroundColor: 'orange'
    }
]
\end{js}

\section{条件渲染}
\verb|v-show|与\verb|v-if|都可以实现条件渲染,但是\verb|v-if|控制的组件节点直接不存在,隐藏的较为彻底。两种都可以使用表达式、布尔值以及Vue管理的属性值作为控制显示的条件。

\begin{html}
<h1 v-show='showAttribute'>Hello, {{ name }}</h1>
<h1 v-show='true'>Hello, {{ name }}</h1>
<h1 v-show='1 === 3'>Hello, {{ name }}</h1>
<hr>
<h1 v-if='1 === 3'>Hello, {{ name }}</h1>
<h1 v-if='true'>Hello, {{ name }}</h1>
<h1 v-if='showAttribute'>Hello, {{ name }}</h1>
\end{html}

\verb|v-if|适用于切换频率较低的场景、不展示的DOM元素直接被移除,其余\verb|v-else-if|使用时不能被打断,需要联合使用,可以配合\verb|template|使用。

\verb|v-show|适用于切换频率较高的场景,不展示的DOM元素不会被移除,仅仅是使用样式隐藏掉。

\section{列表渲染}
列表的渲染可以使用\verb|v-for|指令,其语法格式为:

\verb|v-for="(item, index)" in Object/Array/String/Numbers :key="index"|

\subsection{key的作用与原理}
\textbf{面试题:react、vue中的key有什么作用?(key的内部原理)}

\begin{enumerate}
    \item 虚拟DOM中key的作用:key是虚拟DOM对象的标识,当数据发生变化时,Vue会根据【新数据】生成【新的虚拟DOM】, 随后Vue进行【新虚拟DOM】与【旧虚拟DOM】的差异比较,比较规则如下:
    \begin{enumerate}
        \item 如果旧虚拟DOM中找到了与新虚拟DOM相同的key,则若虚拟DOM中内容没变, 直接使用之前的真实DOM!,否则生成新的真实DOM,随后替换掉页面中之前的真实DOM。
        \item 如果旧虚拟DOM中未找到与新虚拟DOM相同的key,则创建新的真实DOM,随后渲染到到页面。
    \end{enumerate}
    \item 用index作为key可能会引发的问题:
    \begin{enumerate}
        \item 若对数据进行逆序添加、逆序删除等破坏顺序操作,则会产生没有必要的真实DOM更新(界面效果没问题, 但效率低)。
        \item 如果结构中还包含输入类的DOM会产生错误DOM更新(界面有问题)。
    \end{enumerate}
    \item 开发中如何选择key?
    \begin{enumerate}
        \item 最好使用每条数据的唯一标识作为key, 比如id、手机号、身份证号、学号等唯一值。
        \item 如果不存在对数据的逆序添加、逆序删除等破坏顺序操作,仅用于渲染列表用于展示,使用index作为key是没有问题的。
    \end{enumerate}
\end{enumerate}
\subsection{列表过滤}
对于一串字符串,使用\verb|indexOf|搜索,直接搜索空字符串也是可以返回结果的。

\begin{js}
watch: {
    keyWord: {
        immediate: true,
        handler(val) {
            this.filterPersonArr = this.personArr.filter((value) => {
                return value.name.indexOf(val) !== -1
            })
        }
    }
}
\end{js}

对于这个功能,计算属性也可以实现,相比之下更加简单:

\begin{js}
computed: {
    filterPersons() {
        return this.personArr.filter((value) => {
            return value.name.indexOf(this.keyWord) !== -1
        })
    }
}
\end{js}

\subsection{列表排序}
\begin{js}
computed: {
    filterPersons() {
        const arr = this.personArr.filter((value) => {
            return value.name.indexOf(this.keyWord) !== -1
        })
        if (this.sortType) {
            arr.sort((p1, p2) => {
                return this.sortType === 1 ? p1.age - p2.age : p2.age - p1.age
            })
        }
        return arr
    }
}
\end{js}
\subsection{数据监测\ding{72}\ding{72}\ding{72}\ding{72}}
vue会监视data中所有层次的数据。对对象和数组的数据监视有所不同。

监视对象中的数据,通过setter实现监视,且要在new Vue时就传入要监测的数据:
\begin{enumerate}
    \item 对象中后追加的属性,Vue默认不做响应式处理
    \item 如需给后添加的属性做响应式,请使用如下API:
    \begin{enumerate}
        \item \verb|Vue.set(target,propertyName/index,value)|
        \item \verb|vm.$set(target,propertyName/index,value)|
    \end{enumerate}
\end{enumerate}

监视数组中的数据,通过包裹数组更新元素的方法实现,本质就是做了两件事:
\begin{enumerate}
    \item 调用原生对应的方法对数组进行更新
    \item 重新解析模板,进而更新页面
    \item 在Vue修改数组中的某个元素一定要用如下方法:
    \begin{enumerate}
        \item push()、pop()、shift()、unshift()、splice()、sort()、reverse()
        \item \verb|Vue.set()| 或 \verb|vm.$set()|
    \end{enumerate}
\end{enumerate}

\warning{
    Vue.set()和vm.\$set()不能给vm或vm的根数据对象 添加属性
}

\section{表单数据收集}
对于表单搜集的数据,有三种主要的区别:
\begin{itemize}
\item \verb|<input type="text">|,输入文本,则v-model收集的是value值,用户输入的直接是value值;
\item \verb|<input type="radio">|,单选框,则v-model收集的是value值,但是需要给每个标签指定value值;
\item \verb|<input type="text">|,复选框,如果没有配置value的属性,那么收集的是复选框是否被选中(i.e. checked值,勾选或者没勾选,布尔值)。如果配置了value的属性,那么如果v-model的初始值是非数组,收集的就是checked值,数组情况下则收集的是有value组成的数组。
\end{itemize}

\notes{
    v-model有三个事件修饰符:
    \begin{dinglist}{42}
        \item lazy:失去焦点再收集数据(可用于文本框textarea的数据收集)
        \item number:输入字符串转为有效的数字;
        \item trim:输入首位空格过滤;
    \end{dinglist}
}
\section{过滤器}