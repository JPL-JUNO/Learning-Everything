\chapter{Vue基础}
\section{初识Vue}

\section{模板语法}
Vue模板语法有2大类:
\begin{enumerate}
    \item 插值语法:用于解析标签体内容,主要写法是:\verb|{{xxx}}|,\verb|xxx|是JS表达式,且可以直接读取到data中的所有属性。
    \item 指令语法:用于解析标签(包括:标签属性、标签体内容、绑定事件......)。例如:\verb|v-bind:href="xxx"|,\verb|xxx|是JS表达式,且可以直接读取到data中的所有属性。
\end{enumerate}
\section{数据绑定}

Vue中有两种数据绑定方式:
\begin{enumerate}
    \item 单向数据绑定(v-bind):数据只能从data流向页面。
    \item 双向数据绑定(v-model):数据不仅能从data流向页面,还可以从页面流向data。双向绑定一般都应用在表达类元素上,如input、select等,\verb|v-model: value|可以简写为:\verb|v-model|,因为\verb|v-model|默认收集的就是value的值。
\end{enumerate}
\begin{html}
<div id="root">
    <input type="text" v-bind:value="name"><br />
    <input type="text" v-model:value="name">
    <input type="text" v-model="name">
</div>
\end{html}

\section{el与data的两种写法}
\begin{itemize}
    \item el有2种写法:
    \begin{enumerate}
        \item new Vue配置el属性
        \item 先创建Vue实例,然后通过\verb|vm.$mount('#root')|指定el的值。
    \end{enumerate}
    \item data有2种写法:
    \begin{enumerate}
        \item 对象式
        \item 函数式
    \end{enumerate}
\end{itemize}

一个重要的原则:由Vue管理的实例,一定不要写箭头函数,一旦写了箭头函数,this指针就不再是Vue实例了。

\section{理解MVVM模型}

\section{事件处理}
事件的基本使用:
\begin{enumerate}
    \item 使用\verb|v-on:xxx| 或 \verb|@xxx|绑定事件,其中\verb|xxx|是事件名;
    \item 事件的回调需要配置在methods对象中,最终会在vm上;
    \item methods中配置的函数,不要用箭头函数!否则this就不是vm了(变成了window);
    \item methods中配置的函数,都是被Vue所管理的函数,this的指向是vm 或 组件实例对象;
    \item \verb|@click="demo"| 和 \verb|@click="demo($event)"| 效果一致,但后者可以传参;
\end{enumerate}

\subsection{事件修饰符}
主要由以下6种事件修饰符,用于操作对事件的处理。
\begin{enumerate}
    \item prevent:阻止默认事件(常用);
    \item stop:阻止事件冒泡(常用);
    \item once:事件只触发一次(常用);
    \item capture:使用事件的捕获模式;
    \item self:只有event.target是当前操作的元素时才触发事件;
    \item passive:事件的默认行为立即执行,无需等待事件回调执行完毕;
\end{enumerate}

事件修饰符可以连续写,比如\verb|@click.stop.prevent|,先阻止冒泡,然后阻止默认事件。

\subsection{键盘事件}
Vue中常用的按键别名:回车(enter)、删除(delete, 捕获“删除”和“退格”键)、退出(esc)、空格(space)、换行(tab, 特殊,必须配合keydown去使用)、上(up)、下(down)、左(left)、右(right)。

Vue未提供别名的按键,可以使用按键原始的key值去绑定,但注意要转为kebab-case(短横线命名)。

系统修饰键(用法特殊):ctrl、alt、shift、meta
\begin{enumerate}
    \item 配合keyup使用:按下修饰键的同时,再按下其他键,随后释放其他键,事件才被触发。
    \item 配合keydown使用:正常触发事件。
\end{enumerate}

此外,也可以使用keyCode去指定具体的按键(不推荐),\verb|Vue.config.keyCodes.customName = keyCode|,可以去定制按键别名。

\begin{html}
<div id="root">
    <h1>Hello, {{ name }}</h1>
    <input type="text" placeholder="enter for hint information" @keyup.enter="showInfo">
</div>
\end{html}

\begin{html}
<script>
    new Vue({
        el: "#root",
        data: {
            name: 'Vue2-3'
        },
        methods: {
            showInfo(event) {
                // if (event.keyCode !== 13) return
                console.log(event.target.value)
            }
        }
    })
</script>
\end{html}
\section{计算属性与监视属性}
\subsection{计算属性}
需要使用的属性不存在,要通过vm实例已有的属性(Property)计算得来,底层借助了\verb|Object.defineproperty|方法提供的\verb|getter|与\verb|setter|。

get函数执行的时刻为以下2种:
\begin{enumerate}
    \item 初次读取时会执行一次;
    \item 当依赖的数据发生改变时会被再次调用;
\end{enumerate}

尽管使用methods方式以及插值方式都可以实现,但是计算属性由内部缓存机制(复用),调试方便。

\begin{enumerate}
    \item 计算属性最终会出现在vm上,直接读取使用即可。
    \item 如果计算属性要被修改,那必须写set函数去响应修改,且set中要引起计算时依赖的数据发生改变。
\end{enumerate}

\begin{js}
computed: {
    fullName: {
        get() {
            return this.firstName + '-' + this.lastName
        },

        set(value) {
            console.log('set', value);
            const arr = value.split('-');
            this.firstName = arr[0];
            this.lastName = arr[1];
        }
    }
}
\end{js}

如果不考虑修改计算属性,那么get的计算属性可以简写为:
\begin{js}
computed: {
    fullName() {
        return this.firstName + ' ' + this.lastName
    }
}
\end{js}
\subsection{监视属性}
监视属性(watch),当被监视的属性变化时, 回调函数自动调用, 进行相关操作,监视的属性必须存在,才能进行监视。监视有2种写法:
\begin{enumerate}
    \item new Vue时传入watch配置
    \begin{js}
watch: {
    isHot: {
        immediate: true,
        handler(newValue, oldValue) {
            console.log('isHot was moidified', newValue, oldValue)
        }
    }
}
    \end{js}
    \item 通过\verb|vm.$watch|监视
    \begin{js}
vm.$watch('isHot', {
    immediate: true,
    handler(newValue, oldValue) { 
        console.log('isHot被修改了', newValue, oldValue)
    }
})      
    \end{js}
\end{enumerate}

Vue中的watch默认不监测对象内部值的改变(一层、最外层),配置\verb|deep: true|可以监测对象内部值的改变(内层)。

Vue自身可以监测对象内部值的改变,但Vue提供的watch默认不可以,使用watch时根据数据具体结构,决定是否使用深度监测。

监视多层级中某个属性的变化:
\begin{js}
"numbers.a": {
    handler() {
        console.log('a was modified')
    }
}
\end{js}

监测层级中任一属性值的改变:
\begin{js}
numbers: {
    deep: true,
    handler() {
        console.log('numbers were modified')
    }
}
\end{js}